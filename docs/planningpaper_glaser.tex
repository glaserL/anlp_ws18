\documentclass[11pt,a4paper]{article}
\usepackage[hyperref]{acl2018}
\usepackage{times}
\usepackage{latexsym}

%\aclfinalcopy % Uncomment this line for the final submission
%\def\aclpaperid{***} %  Enter the acl Paper ID here

%\setlength\titlebox{5cm}
% You can expand the titlebox if you need extra space
% to show all the authors. Please do not make the titlebox
% smaller than 5cm (the original size); we will check this
% in the camera-ready version and ask you to change it back.


\usepackage{graphicx} % for imports
\usepackage{subcaption} % for figures
\usepackage{cleveref}
\graphicspath{{img/}}
\usepackage{pgfplots} % plotting pgf format directly
% citations

\title{Planning Paper}

\author{Luis Glaser\\
  Matrikelnumber 800140 \\
  Potsdam University \\
  {\tt Luis.Glaser@uni-potsdam.de}
  \\}

\date{}
\usepackage{url}

\begin{document}
\maketitle
\begin{itemize}
\item What is your project about and what approach are you going to take?
\item What tools and/or data will you use?
\item What will you learn from this project? How does the successful imple- mentation of the project help you achieve the learning objectives of the class?
\item What will be your personal role in the team, which skills and what knowl- edge can you contribute?
\end{itemize}


\section{Introduction \& Motivation}

Together with Atreya Shankar and Juliane Hanel I will work on music lyrics. The original motivation was our shared interest in music and the different cultural dimension it expresses. And we were interested whether we would be able to find patterns only in the lyrical data when excluding the acoustic part. Some of these initial questions where things like: Is there a significant difference between rap and pop lyrics? How do different genres differ in their emotionality? How have music lyrics in general evolved over time? As it should be clear by know, we will take a broad approach and try to answer an array of questions all concerned with music lyrics.
This planning paper serves the purpose of structuring those initial questions, in order to keep our research focused. This will happen in k% TODO: FIX
 steps. First what data we will use, because this already will restrict the possible questions we can pose. 

\section{State of the Art}

\section{Method}
\begin{figure}[h]
	\centering
	\resizebox{\linewidth}{!}{
		\includegraphics{img/kaggleplot_bias}
	}
	\caption{Distribution of genre over time in \texttt{kaggle} data}
       \label{fig:kaggle}
\end{figure}
% Data
When first experimenting with our approach, we went with the first available data set from %TODO cite kaggle shit
However, we came across initial flaws, as the dataset was very biased towards a certain year. This can easily be seen in the histogram depicted in \cref{fig:kaggle} we will not use this in the end. We currently have two approaches we could take from here. The first one will be to crawl data by ourselves by interfacing with the genius API.\footnote{\url{https://docs.genius.com}}
This is still in progress and we can't yet decide if this dataset will have less of a bias. The second technique will be to search for further datasets and integrate them into one dataset. This would have multiple disadvantages. First, it's likely that we won't be able to completely get rid of the bias and second will introduce new challenges like merging the data and unifying their formats. This wouldn't provide anything interesting for us to learn. 

\section{Tooling}
\begin{figure}[h!]
	\centering
	\scriptsize
	\begin{tabular}{llllll}
ID & Artist & Title & Year & Genre & Lyrics \\
1 & Rihanna & Diamonds & 2012 & Pop & Shine bright $\ldots$ \\
	\end{tabular}
	\caption{Example entries in \texttt{SQLite} database}
	\label{fig:table}
\end{figure}
For data house keeping we'll use a \texttt{SQLite}\footnote{\url{https://www.sqlite.org}} database which will be shared over \texttt{GitHub} and interfaced via \texttt{python}. Furthermore as mentioned already in our group contract, we will use this \texttt{GitHub} repository for collaboration in general. 

In both cases we will scrutinize % WORT richtig?
our data for any irregularities or unforseen biases and .. mach ma hoit wos mit ean.

One of my responsibilities will be to detect and annotate political sentiment expressed in lyrics. Right now I intend to use a \texttt{Word2Vec} to represent lyrics and extrapolating their political expression from their. However, how to validate my results is still unclear and will need more work in the upcoming weeks. % TOdo cite mikolov paper

\section{Personal Contribution}

I will contribute to this project on multiple different ends. 

I will be responsible of taking care of our database system. In both my undergraduate degrees I had had to work with rather large datasets that kept changing during work. Thus, I am familiar with a few pitfalls concerning data handling. My work should ensure that all of us can work undisrupted and not have to bother a lot with the usual data juggling. 

Furthermore I will take care of annotating political stance % Einstellung?
to the lyrics (if) they are expressed. I wanted to do this within my undergrad thesis in political science, but didn't get around to and was beyond scope for doing it for German data. % TODO: Cite das hard paper

Furthermore, I already did work with quite a few of the annotations that we will create during our analysis. Thus, I should be able to contribute erfahrung in that area and provide help to my fellow teammates. 

After our annotation phase, I will join Atreya and Juliane with their change analysis. I will aim at bringing my traditional engineering background to the table, trying to solidify our results with sceptical thinking. However, I feel Atreya and Juliane have way more experience then me in this field, which I hope to profit of.

\section{Learning Goals}

This brings us to the personal learning goals.
Our project will use quite a few techniques that were not discussed in class. However we hope that using those techniques will give us not only the ability to apply the knowledge and methods we learned in class on another project. 
We also will be able to try out new methods which will help me / us when deciding of the further classes to take during the masters programm. This is especially true for the rather exploratory approach that we intend to take. Each hypothesis can be tested in a fairly short amount of time. Testing multiple highly different hypothesis will give us all the opportunity to try out different methods. 




\bibliographystyle{acl}
\bibliography{bibtex}



\end{document}

